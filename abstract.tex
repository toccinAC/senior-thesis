\begin{abstract}
Large-scale systems are very complex and because of that, it is
very difficult to test them adequately. The current metric to determine if a system
has been thoroughly tested is called coverage. Test coverage as a metric can be a misleading
number that can cause test suites to be ineffective without the creator knowing.
Coverage is used to explain how much of a program is tested, but this does not
reveal whether or not the tests are adequate. This is an industry-wide issue as
many companies will not allow pull requests if the test coverage is any lower
than 100\%\cite{prause2017100}. This limitation can potentially lead to
something called a pseudo-tested method. A pseudo-tested method is a method that
is tested, but which passes regardless of the output of a function. So if a test passes without ensuring that the
output was as intended by the developer, that is a pseudo-tested method because it is believed that
the test is effective. A tool has been created, aptly named Function-Fiasco, that
helps determine how much of the code is adequately tested and provides a
metric that is more representative of the actual conditions of the system. The
tool is based on the idea of fault injection. Software fault injection denotes the artificial insertion, or injection, of faults and error states into a running software system~\cite{feinbube2018software}. The tool allows a function to run, but collects the output, and then
randomizes the output. It then runs the tests associated to the function using randomized output and if the test still passes, it is
a pseudo-tested method. It then produces a metric that is
representative of the behavior of the system. This study also has a completed analysis of this
information to determine if the system was accurate in detecting pseuodo-tested
methods.
\end{abstract}
